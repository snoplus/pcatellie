\documentclass[xcolor=table]{beamer}
\usepackage[utf8]{inputenc}
\usepackage[T1]{fontenc} 
\usepackage[english]{babel} 
\usepackage{graphicx}
\usepackage{comment}
\usepackage{graphicx,wrapfig,lipsum}
\usepackage{hyperref}
\usepackage{tcolorbox}
\hypersetup{
    colorlinks=true,
    linkcolor=blue,
    filecolor=magenta,      
    urlcolor=cyan,
}
\setbeamertemplate{footline}[text line]{%
  \parbox{\linewidth}{\vspace*{-8pt}\today\hfill\insertshortauthor\hfill\insertpagenumber}}{}

%%Defining the ``proposition'' environment
\newtheorem{proposition}{Proposition} 

%%Sets the beamer theme "Cuerna"
\usetheme{Cuerna} 

\usecolortheme{default}
%Available color themes: default, bluesimplex, brick, lettuce

%%Insert the logo
\logo{\includegraphics[width=1.7cm]{plots/logo.jpeg}}

%%Title
\title{TELLIE PCA:\\ Processing \\ Automation}

\author{Michal Rigan\\ %Author
          \texttt{mrigan@snolab.ca}} %e-mail
          
\date{\textbf{Report}\\
\today} %Date or event

\institute{University of Sussex} %%Institution

\begin{document}

{
\setbeamertemplate{footline}{} 
\begin{frame}
  \titlepage %Creates the title page
\end{frame}
}

\begin{frame}{Processing automation - Why}
\begin{itemize}
	\item extracting and validating PCA constants from data is complex...
	\item \textbf{Goal}: streamline (possibly speed up) the process of obtaining the PCA constants from data
	\begin{itemize}
		\item regardless of the method to obtain the data
		\item modular
		\item require minimum human input
		\item provide monitoring
	\end{itemize}
\end{itemize}
\end{frame}

\begin{frame}
\noindent\makebox[\textwidth]{\includegraphics[width=0.491\textwidth]{plots/overview.png}}
\end{frame}

\begin{frame}{Processing automation - What}
\begin{itemize}
	\item validates data is `good enough` for PCA $\rightarrow$ \textit{validate \#1}
	\item fits for required corrections: beamspot fit, fibre direction, angular systematic, injection time $\rightarrow$ \textit{PCA table}
	\item compare these fits and runtime values to previous set (stability) $\rightarrow$ \textit{PCA table}
	\item validates fits are `sensible` $\rightarrow$ \textit{validate \#2}
	\item extracts PCA constants (PCA processor) $\rightarrow$ \textit{PCA constants}
	\item benchmarks the constants against previous set $\rightarrow$ \textit{Benchmarking}
	\item provides monitoring of each step, and between datasets (!) $\rightarrow$ \textit{Monitoring}
\end{itemize}
\end{frame}

\begin{frame}{Processing automation - Validations}
Run series of checks:
\begin{itemize}
	\item Validation \#1:
	\begin{itemize}
		\item correct fibre, number of events (EXTA), passed hits, cuts on PMTs, checks on LPC, run length, frequency
		\item NHit distribution, NHit over time, delays
		\item time of hits over time, \# peaks, PMTs in beamspot, PMT occupancy
		\item PIN, PIN vs NHit, events over subruns, ... (21 total)
	\end{itemize}
	\item Validation \#2:
		\begin{itemize}
		\item for each correction: check mean, rms, min, max 
		\item residual times: distribution, \# peaks, function of angle
		\item evaluate trends (12 total)
	\end{itemize}
	\item this is available on monitoring page (flags $\rightarrow$ bitword)
\end{itemize}
\end{frame}

\begin{frame}{Processing automation - Benchmarking}
\begin{itemize}
	\item compare PCA values (cable delays, TW fit) to previous set
	\item apply these constants to a well understood run
	\item extract the residual hit times distribution
	\item monitor charges: threshold, peak, hhp for QHS \& QHL
\end{itemize}
\end{frame}

\begin{frame}{Processing automation - How}
\begin{itemize}
	\item \textit{simple} $\rightarrow$ only requires to provide a runlist
	\item \textit{modular} $\rightarrow$ master script that spawns subprocesses, individual steps can be (re)run. Also allows for easier changes to modules
	\item \textit{submission platform} $\rightarrow$ can queue processes, submit (up to a limit), monitor their status
	\item \textit{customizable} $\rightarrow$ thresholds (other settings) are loaded from environment (tuning)
	\item \textit{linked} $\rightarrow$ stores data in couchdb, ratdb, redis, provides plots to minard
	\item \textit{regulation} $\rightarrow$ unifies cuts, data checks, event selection, ranges, ...
	\item \textit{evaluative} $\rightarrow$ provides bitwords (flags) for fits / checks
\end{itemize}
\end{frame}

\begin{frame}{Processing automation - Minard}
\noindent\makebox[\textwidth]{\includegraphics[width=1\textwidth]{plots/1.png}}
\end{frame}

\begin{frame}{Processing automation - Minard: PCA processing}
\noindent\makebox[\textwidth]{\includegraphics[width=1\textwidth]{plots/2.png}}
\end{frame}

\begin{frame}{Processing automation - Minard: PCA dataset}
\noindent\makebox[\textwidth]{\includegraphics[width=1\textwidth]{plots/3.png}}
\end{frame}

\begin{frame}{Processing automation - Minard: validation 1}
\noindent\makebox[\textwidth]{
\includegraphics[width=0.26\textwidth]{plots/4.png}
\includegraphics[width=0.49\textwidth]{plots/5.png}}
\end{frame}

\begin{frame}{Processing automation - Minard: validation 1}
\noindent\makebox[\textwidth]{
\includegraphics[width=0.3\textwidth]{plots/6.png}
\includegraphics[width=0.4\textwidth]{plots/7.png}}
\end{frame}

\begin{frame}{Processing automation - Minard: PCA tables}
\noindent\makebox[\textwidth]{
\includegraphics[width=0.45\textwidth]{plots/8.png}
\includegraphics[width=0.45\textwidth]{plots/9.png}}
\end{frame}

\begin{frame}{Processing automation - Minard: PCA table}
\noindent\makebox[\textwidth]{
\includegraphics[width=0.3\textwidth]{plots/10.png}
\includegraphics[width=0.3\textwidth]{plots/11.png}
\includegraphics[width=0.3\textwidth]{plots/12.png}}
\noindent\makebox[\textwidth]{\includegraphics[width=0.8\textwidth]{plots/13.png}}
\end{frame}

\begin{frame}{Processing automation - Minard: PCA processor}
\noindent\makebox[\textwidth]{
\includegraphics[width=0.4\textwidth]{plots/20.png}
\includegraphics[width=0.6\textwidth]{plots/21.png}}
\end{frame}

\begin{frame}{Processing automation - Minard: PCA processor}
\noindent\makebox[\textwidth]{
\includegraphics[width=0.6\textwidth]{plots/22.png}
\includegraphics[width=0.4\textwidth]{plots/23.png}}
\end{frame}

\begin{frame}{Processing automation - Minard: charge monitoring}
\noindent\makebox[\textwidth]{
\includegraphics[width=0.49\textwidth]{plots/24.png}
\includegraphics[width=0.49\textwidth]{plots/25.png}}
\end{frame}

\begin{frame}{Processing automation - Next steps}
\begin{itemize}
	\item missing from monitoring: page for PMT, parsing of log files
	\item comments, descriptions, nicer grouping ...
	\item documentation
	\item tuning of the threshold values
	\item test over datasets
	\item ...
	\item profit (?)
\end{itemize}
\end{frame}

\end{document}